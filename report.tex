\documentclass[12pt]{article}
\usepackage{amsmath}
\usepackage{amssymb}
\usepackage{graphicx}
\usepackage{geometry}
\geometry{
  letterpaper,
  total={7.5in, 10in},
  left=0.5in,
  top=0.5in,
}

\begin{document}
\title{SURF Detector Characterization Report:}
\author{}
\maketitle

\section{$^{228}$Th Source Measurements}

\begin{figure}[!htb]
  \centering
  \includegraphics[width=0.6\textwidth]{energy__Th.pdf}
  \caption{The calibrated energy spectrum from the trapezoidal filter maximum and fixed time pickoff with and without the charge trapping correction.  The spectrum shown in green applies the $A/E$ cut to the spectrum with the fixed time pickoff.\label{fig:energy_Th}}
\end{figure}

\begin{figure}[!htb]
  \centering
  \includegraphics[width=0.8\textwidth]{peakfit__Th.pdf}
  \caption{Fits to six peaks in the $^{228}$Th spectrum.  Note that the calibrated energy scale is approximate.\label{fig:peakfit_Th}}
\end{figure}

\begin{figure}[!htb]
  \centering
  \includegraphics[width=0.6\textwidth]{elin__Th.pdf}
  \caption{Top: the peak centroid positions from Figure~\ref{fig:peakfit_Th} as a function of the true energy.  Bottom: residuals from the linear fit in the top panel.\label{fig:elin_Th}}
\end{figure}

\begin{equation}
  \label{eq:fwhm}
  \text{FWHM}(E) = \sqrt{\Gamma^2_n+\Gamma^2_FE+\Gamma^2_qE^2}
\end{equation}

\begin{figure}[!htb]
  \centering
  \includegraphics[width=0.6\textwidth]{fwhm__Th.pdf}
  \caption{Top: the FWHM as a function of energy from the fits in Figure~\ref{fig:peakfit_Th}, including the exponentially modified Gaussian tails.  The red curve shows a fit to Equation~\ref{eq:fwhm}.  The vertical green line shows the projected FWHM at $Q_{\beta\beta}$.  Bottom: residuals from the fit in the top panel.\label{fig:fwhm_Th}}
\end{figure}

\begin{figure}[!htb]
  \centering
  \includegraphics[width=\textwidth]{psaE__Th.pdf}
  \caption{The AvsE, A/E, and DCR parameters as a function of energy.  The AvsE and A/E parameters are approximately calibrated to 90\% acceptance of the DEP from the 2615~keV peak at the parameter value of -1.  The labels above the AvsE and A/E panels indicate the DEP and SEP acceptance.\label{fig:psaE_Th}}
\end{figure}

\begin{figure}[!htb]
  \centering
  \includegraphics[width=\textwidth]{psa__Th.pdf}
  \caption{The AvsE, A/E, and DCR parameter distributions in a 10~keV wide region centered on the 2615~keV peak.\label{fig:psa_Th}}
\end{figure}

\end{document}
